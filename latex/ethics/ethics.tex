\section{User Study}
We conducted a user study to evaluate our simulator, titled "A Virtual Volumetric Screen User Study." This study followed the ethical guidelines and approval process as outlined by the Science Engineering Technology Research Ethics Committee at Imperial College London. The process was reviewed by the Research Governance and Integrity Team and the head of the Computing Department. We received approval on 2nd May 2024 and conducted the study between the 1st and 5th of June.

\subsection{Human Participants}
Since our study involved human participants, we ensured that it was conducted ethically and in accordance with relevant guidelines. We adhered to the Equality Act 2010 \cite{participation_equality_nodate} to avoid excluding any participants. Additionally, we took care to avoid involving participants who might feel pressured to participate or influenced by the researchers. Prior to the commencement of the study, participants were provided with an information sheet and a consent form.

\subsection{Data Collection}
During data collection, we complied with local regulations, including the General Data Protection Regulation (GDPR) \cite{EuropeanParliament2016a}. All data were securely stored on a computer located on campus, accessible only to the researchers, or on Imperial's secure cloud network. We ensured that data presented in our report were anonymized. The data will be retained until 30th June 2034.

\section{Volumetric Simulator}
\subsection{Military Applications}
Although this technology could theoretically be used for military applications, we believe it is unlikely to be employed for such purposes. If it were to be used, it would not be for direct combat and would pose no more danger than other existing technologies.

\subsection{Copyright Limitations}
\subsubsection{Open Source}
We utilise several open-source libraries and tools in our project. The Azure Kinect SDK is licenced under the MIT licence. We use the Nix package manager, licenced under the LGPL-2.1 licence, and the Nixpkgs repository, which is also under the MIT licence. The \texttt{dlib} library is used under the Boost Software Licence 1.0 (BSL-1.0). The OpenGL library is employed under its open-source licence for the Sample Implementation (SI). We also utilise the GLFW library, licenced under the zlib licence, and the GLM library, licenced under the MIT licence. Additionally, we use OpenCV, which is licenced under the Apache Licence, and MediaPipe, which is released under the Apache Licence 2.0. The tinyobjloader is used under the MIT licence.

\subsubsection{Proprietary Software}
We also use proprietary software in our project. This includes Microsoft's proprietary depth engine, designed for use with the Azure Kinect SDK, which is not open source. Additionally, we use CUDA and related libraries, which are proprietary software developed by NVIDIA.
