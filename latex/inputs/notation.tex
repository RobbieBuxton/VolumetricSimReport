% Here, you can define your own macros. Some examples are given below.

% \newcommand{\R}[0]{\mathds{R}} % real numbers
% \newcommand{\Z}[0]{\mathds{Z}} % integers
% \newcommand{\N}[0]{\mathds{N}} % natural numbers
% \newcommand{\C}[0]{\mathds{C}} % complex numbers
% \renewcommand{\vec}[1]{{\boldsymbol{{#1}}}} % vector
% \newcommand{\mat}[1]{{\boldsymbol{{#1}}}} % matrix

\newcommand{\tocite}{{\color{red} \small ToCite }} % placeholder for later citation
\newcommand{\todo}{{\color{red} \small TODO }} % placeholder for later citation

\lstdefinestyle{tree}{
literate=
  {├}{{\smash{\raisebox{-1ex}{\rule{1pt}{\baselineskip}}}\raisebox{0.5ex}{\rule{1ex}{1pt}}}}1
{─}{{\raisebox{0.5ex}{\rule{1.5ex}{1pt}}}}1
{└}{{\smash{\raisebox{0.5ex}{\rule{1pt}{\dimexpr\baselineskip-1.5ex}}}\raisebox{0.5ex}{\rule{1ex}{1pt}}}}1
}

\newcommand{\nixstore}[2][1]{\textcolor{Purple}{\texttt{/nix/store/}}\textcolor{RoyalBlue}{\texttt{#1}-}\textcolor{Orange}{\texttt{#2}}}

\newcommand{\mynewminted}[3]{%
  \newminted[#1]{#2}{#3}%
  \tcbset{myminted/#1/.style={minted language=#2,minted options={#3}}}}


%Programming Languages
\mynewminted{nix}{nix}{
  breaklines,
  linenos,
  autogobble,
  numbersep=2mm,
  fontsize=\footnotesize,
  frame=none}

\mynewminted{cpp}{cpp}{
  breaklines,
  linenos,
  autogobble,
  numbersep=2mm,
  fontsize=\footnotesize,
  frame=none}

\mynewminted{shell}{shell}{
  breaklines,
  autogobble,
  fontsize=\footnotesize,
  frame=none}

 % Define a global counter for figures
\newcounter{globalFigureCounter}

\newtcbinputlisting[use counter=globalFigureCounter, number within=section,list inside=mypyg]{\codeBoxFile}[4][]{%
  center,
  listing engine=minted,
  listing only,
  title={{\color{Black} \textbf{Listing \thetcbcounter:}} {\small #4}},
  % list entry={\protect\numberline{\thetcbcounter}#3},
  listing file={#3},
  enhanced jigsaw,
  breakable,
  colframe = Apricot!25,
  colback  = Apricot!10,
  coltitle = Apricot!20!black,
  drop fuzzy shadow,
  before skip = 20pt,
  after skip = 20pt,
  myminted/#2,
  #1}

%Boxes

\newtcbox[use counter=globalFigureCounter, number within=section]{\pictureBox}[2][]
{
  enhanced jigsaw,
  nobeforeafter,
  colframe = Apricot!25,
  colback= Apricot!25,
  coltitle = Apricot!20!black,
  drop fuzzy shadow,
  boxsep=3pt,
  left=0pt,
  right=0pt,
  top=0pt,
  bottom=0pt,
  title    = {{\color{Black} \textbf{Figure \thetcbcounter:}} {\small #2}},
  #1,
}


\newtcolorbox[use counter=globalFigureCounter, number within=section]{figureBox}[2][]
{
  center,
  enhanced jigsaw,
  breakable,
  colframe = Apricot!25,
  colback  = Apricot!10,
  coltitle = Apricot!20!black,
  drop fuzzy shadow,
  title    = {{\color{Black} \textbf{Figure \thetcbcounter:}} {\small #2}},
  before skip = 20pt,
  after skip = 20pt,
  before upper={\centering \color{Gray!20!black}},
  #1,
}

%Invisible box

\newtcolorbox{invisBox}{
  center,
  enhanced jigsaw,
  breakable,
  before skip=20pt,
  after skip=20pt,
  left=0pt,
  right=0pt,
  top=0pt,
  bottom=0pt,
  colframe=white, % Frame color
  colback=white,  % Background color
  opacityframe=0, % Frame opacity
  opacityback=0   % Background opacity
}