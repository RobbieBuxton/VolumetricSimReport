\section{Demonstrated Functionality}
\subsection{Eye Tracker}
One of the key components of our volumetric display simulator is the eye tracker. It is responsible for tracking the position and orientation of the user's head. This is used to render the volumetric display from the correct perspective. To properly evaluate our simulator we must first evaluate the quality of the eye tracker. The key metrics to track would be at different camera input resolutions (we can vary the resolution by pyramiding down) what is the frame rate the tracker can run (bounded by azure kinect's maximum fps of 30). What percentage of the time can it detect an eye during an example input and what are the maximum orientations of a face that it can detect an eye. We can also compare the accuracy of the eye tracker to other eye trackers.

\subsection{Renderer}
The render is responsible for rendering the volumetric display from the correct perspective. To properly evaluate our simulator we prove that rendered scene is accurate. We can do this by comparing the rendered scene to real scene we are trying to simulate. We can do this by scanning a real scene with a depth camera and then rendering the same scene with our simulator and comparing the two from the same perspective. There are currently no open source volumetric display simulators that we can compare our simulator to.

\subsection{Reproducibility}
The purpose of building this project with nix was to provide a reproducible platform for conducting HCI research into volumetric displays. We need to show that we can easily build this project on a variety of platforms and that the results of any experiments conducted one platform can be completely reproduced on another platform (i.e by recording the output of the tracker camera and re-using it on a different machine). 

\section{Success Criteria}
If the simulator is able to be effectively used to conduct novel HCI research into volumetric displays then we will know we have succeeded. 

\section{Novel Contributions}
Once this project is completely we expect to have made the following novel contributions:
\begin{itemize}
    \item A \textbf{volumetric display simulator} that is: Multi-platform, Lightweight, Cheap, Simple, and Reproducible.
    \item A \textbf{user experiment} that compares the effectiveness of using hand tracking to interact directly with an ethereal/incorporeal volumetric display compared to a via tele-operation with a corporeal/tangible display.
\end{itemize}