\section{Motivations}

Volumetric Displays are a new and exciting technology that has the potential to revolutionize the way we interact with computers. They are a type of 3D display that can be viewed from any angle without the need for special glasses by multiple people simultaneously. \tocite These displays differ from a virtual reality experience in that they are not immersive, but rather they are a window into a virtual world (See Fig~\ref{fig:volumetric-display-examples}). There is a not a consensus what the best way to build a volumetric display is and as I cover in the background section there are many approaches being attempted by research groups both academic and industrial. 

\begin{figureBox}[label={fig:volumetric-display-examples}]{Two different volumetric displays}
    \begin{minipage}[t]{0.48\textwidth}
   
      \small {a) Example one.}
    \end{minipage}\hfill
    \begin{minipage}[t]{0.48\textwidth}

      \small {b) Example two.}
    \end{minipage}
\end{figureBox}

It is difficult to conduct human-computer interaction (HCI) research into volumetric displays because these devices are not widely available, expensive to manufacture, and have high bandwidth requirements. This makes it difficult to conduct user studies and experiments. People have created virtual simulations of volumetric displays to try and solve this problem, \tocite but these solutions are often complicated and difficult and expensive to replicate.

\section{Objectives}

With this paper, we aim to provide a cheap, multi-platform, lightweight and simple platform for simulating volumetric displays. We hope that this will enable researchers to conduct HCI research into volumetric displays without the need for expensive hardware. We aim to make the following contributions:
\subsection{Volumetric Simulator}
We plan to create a platform for simulating volumetric displays that is:
\begin{itemize}
    \item \textbf{Multi-platform}: We package our platform in nix \cite{dolstra2004nix} which allows it to be run on any platform that supports nix  (Linux, macOS, and (Windows through WSL \tocite)) on a large range of nix supported hardware \tocite by running a single line of code \texttt{sudo nix run github:RobbieBuxton/VolumetricSim}.
    
    \item \textbf{Lightweight}: By using simple rending algorithms in OpenGL to render the volumetric display our software is relatively computationally cheap to run compared to typical rendering engines that might be used for HCI research like Unity \tocite. 
    
    \item \textbf{Cheap}: By relying just on a depth camera and a normal monitor our software requires minimal hardware to run. This makes it cheap to run and easy to replicate.
    
    \item \textbf{Simple}: We have designed our software to be as simple as possible by taking advantage of the nix package manager to handle all the dependencies and by depending on simple external libraries like \texttt{dlib} \tocite to handle more complicated tasks like face detection. This makes it easy to replicate and modify.
    
    \item \textbf{Reproducible:} By building with nix we can guarantee that any experiments conducted using the simulator will be completely reproducible.
\end{itemize}

\subsection{User experiment}
We plan to conduct an HCI user study to demonstrate the utility of our volumetric simulation platform. We will conduct a user study to compare the effectiveness of using hand tracking to interact directly with an ethereal/incorporeal volumetric display compared to a via tele-operation with a corporeal/tangible display (See Fig~\ref{fig:user experiment}). 

\begin{figureBox}[label={fig:user experiment}]{User experiment}
    % \includegraphics[width = 0.5\linewidth]{}
\end{figureBox}